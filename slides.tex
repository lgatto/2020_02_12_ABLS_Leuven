\documentclass{beamer}

%-----------------------
%% This template is based on a template developed by the SMCS
%% (Statistical Methodology and Computing Service) at UCLouvain

%-----------------------
% pour faire les slides papiers :
% \documentclass[handout, compress]{beamer}
% \usepackage{pgfpages}
% % 4 par page
% \pgfpagesuselayout{4 on 1}[a4paper,border shrink=5mm, landscape]
% % 2 par page
% \pgfpagesuselayout{2 on 1}[a4paper,border shrink=5mm]

\usetheme{UCL2018}

\usepackage[utf8]{inputenc}
\usepackage[T1]{fontenc}
\usepackage{lmodern}
\usepackage{xspace}
\usepackage{float}
\usepackage{graphicx}
\usepackage{amssymb}
\usepackage{hyperref}
\usepackage{ragged2e}
\usepackage[authoryear, round]{natbib}

%% For slides in French
%% \usepackage[french]{babel}
%% \usepackage[cyr]{aeguill}

\usepackage{lipsum}

\theoremstyle{example}
\newtheorem{examplef}{Example}
\newtheorem{examplesf}{Examples}
\newcommand{\ei}{\end{itemize}}

\usepackage{xmpincl}

\newcommand{\sidebysidecaption}[4]{%
\RaggedRight%
  \begin{minipage}[t]{#1}
    \vspace*{0pt}
    #3
  \end{minipage}
  \hfill%
  \begin{minipage}[t]{#2}
    \vspace*{0pt}
    #4
\end{minipage}%
}


\title{Probabilistic mapping  of the sub-cellular proteome}
\author[]{Laurent Gatto}
\date{\today}
\institute[]{CBIO, de Duve Institute, UCLouvain}


\begin{document}

\pdfinfo {/Author(Laurent Gatto - UCLouvain)}


%-----------------------------------------------
% Title page
%-----------------------------------------------

\begin{frame}[plain]
\titlepage
\end{frame}


\begin{frame}%[noframenumbering]
%\thispagestyle{empty}

% Logo UCL a gauche
%% \begin{tikzpicture}
%%   \useasboundingbox (0,0) rectangle(\the\paperwidth,1);
%%   \node[inner sep=0pt] at (1.7,.5) {\includegraphics[width=.33\textwidth]{UCL_2018}};
%%  \end{tikzpicture}

\justify

{\small \textbf{Abstract:} In biology, localisation is function -
  understanding the sub-cellular localisation of proteins is paramount
  to comprehend the context and full extend of their
  functions. Shotgun mass spectrometry-based spatial proteomics method
  are orthogonal to widely used targeted microscopy-based assay. In
  conjunction with contemporary machine learning, the former enable to
  build proteome-wide protein localisation maps, informing us on the
  location of thousands of proteins. When studying these proteome-wide
  spatial maps, one can learn that while some proteins can be found in
  a single location within a cell, up to half of the proteins may
  reside in multiple locations, can dynamically re-localise, or reside
  within an unknown functional compartment, leading to considerable
  uncertainty in associating proteins to their sub-cellular
  location. Recent Bayesian modelling approaches enable us to mine
  these data, and in particular the dynamic fraction of the spatial
  proteome, in much greater depth. We are now in a position to (1)
  probabilistically model protein localisation as well as quantify the
  uncertainty in the location assignments, and (2) compute a
  probability for, and quantify uncertainty in, whether a protein is
  differentially localised upon cellular perturbation. These
  computational approaches lead to better and more trustworthy
  biological interpretation of these rich spatial proteomics data.  }


\end{frame}


\begin{frame}{Acknowledgements}

  \begin{itemize}

  \item \textcolor{Blue}{Mr Oliver Crook} (Cambridge)

  \item \textcolor{Blue}{Dr Lisa Breckels} (Cambridge)

  \end{itemize}


\end{frame}


%-----------------------------------------------
% Table of content
%-----------------------------------------------

\AtBeginSection[]{
  \setbeamercolor{background canvas}{bg=UCLblue2}
  \setbeamercolor{section in toc}{fg=UCLblue}
  \setbeamercolor{subsection in toc}{fg=UCLblue}
  \setbeamerfont{section in toc}{size=\large}

  \mode<handout>{
    \setbeamercolor{background canvas}{bg=white}
    \setbeamercolor{section in toc}{fg=UCLblue}
    \setbeamercolor{subsection in toc}{fg=UCLblue}
  }

  \begin{frame}[plain]
    \frametitle{Outline}
    \tableofcontents[currentsection,hideothersubsections]
  \end{frame}

  \setbeamercolor{background canvas}{bg=white}
}


%-----------------------------------------------
% Content
%-----------------------------------------------

\section{Spatial proteomics}


\begin{frame}{Cell organisation - \textbf{localisation is function}}
  \begin{center}
    \includegraphics[width=.8\linewidth]{figs/Animal_cell_structure.png} \\
    \textbf{\textcolor{Blue}{Spatial proteomics}} is the systematic
    study of protein localisations.
  \end{center}

  \begin{center}
    \textbf{Localisation -- re-localisation -- mis-localisation}
  \end{center}
  
  \tiny Image from Wikipedia
  \url{http://en.wikipedia.org/wiki/Cell_(biology)}.  
\end{frame}


\begin{frame}{}

  \begin{columns}
    \begin{column}{0.5\textwidth}

      \textbf{Explorative/discovery approaches},
      \textcolor{Blue}{steady-state \textbf{global localisation maps}}
      (as opposed to microscopy-based targeted approaches).

      \bigskip
      
      \small{

        \textbf{Density gradient}: PCP \citep{Dunkley:2006}, LOPIT
        \citep{Foster2006}, hyperLOPIT
        \citep{Christoforou:2016,Mulvey:2017} and \\

        \textbf{Differential centrifugation} \cite{Itzhak:2016},
        LOPIT-DC \citep{Geladaki:2018}.

      }
     
      \bigskip
      
      
    \end{column}
    \begin{column}{0.5\textwidth}
      \includegraphics[width=.78\linewidth]{figs/workflow_primary.pdf}
    \end{column}    
  \end{columns}
  
\end{frame}

\subsubsection*{The data}
\label{sec:data}

\begin{frame}{Quantitation data}
  \begin{center}
    \begin{tabular}{|l|llll|}
      \hline
      & Fraction$_{\text{1}}$ & Fraction$_{\text{2}}$ & \ldots{} & Fraction$_{\text{L}}$ \\
      \hline
      {\bf x}$_{\text{1}}$ & $x_{\text{1,1}}$ & $x_{\text{1,2}}$ & \ldots{} & $x_{\text{1,L}}$ \\
      {\bf x}$_{\text{2}}$ & $x_{\text{2,1}}$ & $x_{\text{2,2}}$ & \ldots{} & $x_{\text{2,L}}$ \\
      {\bf x}$_{\text{3}}$ & $x_{\text{3,1}}$ & $x_{\text{3,2}}$ & \ldots{} & $x_{\text{3,L}}$ \\
      \vdots & \vdots & \vdots & \vdots & \vdots \\
      {\bf x}$_{\text{i}}$ & $x_{\text{i,1}}$ & $x_{\text{i,2}}$ & \ldots{} & $x_{\text{i,L}}$ \\
      \vdots & \vdots & \vdots & \vdots & \vdots \\
      {\bf x}$_{\text{N}}$ & $x_{\text{N,1}}$ & $x_{\text{N,2}}$ & \ldots{} & $x_{\text{N, L}}$ \\
      \hline
    \end{tabular}
  \end{center}
\end{frame}

\begin{frame}{Quantitation data and organelle markers}
  \begin{center}
    \begin{tabular}{|l|llll||l|}
      \hline
      & Fraction$_{\text{1}}$ & Fraction$_{\text{2}}$ & \ldots{} & Fraction$_{\text{L}}$ & markers\\
      \hline
      {\bf x}$_{\text{1}}$ & $x_{\text{1,1}}$ & $x_{\text{1,2}}$ & \ldots{} & $x_{\text{1,L}}$ & unknown \\
      {\bf x}$_{\text{2}}$ & $x_{\text{2,1}}$ & $x_{\text{2,2}}$ & \ldots{} & $x_{\text{2,L}}$ & \textcolor{Red}{$loc_{1}$}\\
      {\bf x}$_{\text{3}}$ & $x_{\text{3,1}}$ & $x_{\text{3,2}}$ & \ldots{} & $x_{\text{3,L}}$ & unknown \\
      \vdots & \vdots & \vdots & \vdots & \vdots & \vdots \\
      {\bf x}$_{\text{i}}$ & $x_{\text{i,1}}$ & $x_{\text{i,2}}$ & \ldots{} & $x_{\text{i,L}}$ & \textcolor{Blue}{$loc_{k}$}\\
      \vdots & \vdots & \vdots & \vdots & \vdots & \vdots\\
      {\bf x}$_{\text{N}}$ & $x_{\text{N,1}}$ & $x_{\text{N,2}}$ & \ldots{} & $x_{\text{N, K}}$ & unknown \\
      \hline
    \end{tabular}
  \end{center}
\end{frame}


%-----------------------------------------------
% New section
%-----------------------------------------------

\section{Data analysis}


\begin{frame}{Visualisation}
  \begin{figure}
    \centering
    \includegraphics[width=.6\linewidth]{figs/F04-analyses.pdf}
    \caption{From \cite{Gatto:2010}, \textit{Arabidopsis thaliana} data
      from \cite{Dunkley:2006}}
  \end{figure}
\end{frame}

\begin{frame}{Quality control}
  \begin{figure}
    \includegraphics[width=.3\linewidth]{figs/F04-analyses.pdf}
    \includegraphics[width=.3\linewidth]{figs/F04-analyses.pdf}
    \includegraphics[width=.3\linewidth]{figs/F04-analyses.pdf}
    \caption{Assessing sub-cellular resolution in spatial proteomics
      experiments \citep{Gatto:2018}}
  \end{figure}
\end{frame}


\begin{frame}{Problem statement: classification}
  \begin{figure}[h]
    \centering
    \includegraphics[width=\linewidth]{figs/hyperlopit-class.pdf}
    \caption{Support vector machines classifier (after 5\% FDR
      classification cutoff) on the embryonic stem cell data from
      \cite{Christoforou:2016}.}
  \end{figure}
\end{frame}

%-----------------------------------------------
% New section
%-----------------------------------------------

\section{Computational challenges}


\begin{frame}{Computational challenges}

  \begin{itemize}
  \item Visualisation (cluster, unsupervised learning)
  \item Classification (supervised learning)
  \item \textbf{Novelty detection} (semi-supervised learning)
  \item Data integration (transfer learning)
  \item \textbf{Unvertainty quantification}
  \item \textbf{Multi-localisation}
  \item \textbf{Spatial dynamics}
  \end{itemize}
  \centering

  \bigskip

  {\Large To uncover and understand biology}
\end{frame}


%-----------------------------------------------
% New section
%-----------------------------------------------

\section{Novelty detection}


\begin{frame}{Importance of annotation}

  \begin{columns}[t]
    \begin{column}[T]{0.43\textwidth}
      \begin{centering}
        %% \includegraphics[width=1\linewidth]{figs_all/tan2009r1org.pdf}
      \end{centering}
    \end{column}
    \begin{column}[T]{0.56\textwidth}
      %% \includegraphics[width=1\linewidth]{figs_all/Animal_cell_structure.png}
    \end{column}
  \end{columns}
  Incomplete annotation, and therefore lack of training data, for
  many/most organelles. \textit{Drosophila} data from \cite{Tan2009}.
\end{frame}

\begin{frame}{Semi-supervised learning: novelty detection}
  \begin{figure}
    %% \includegraphics[width=.48\linewidth]{figs_all/tan2009r1org.pdf}
    %% \includegraphics[width=.5\linewidth]{figs_all/pdres2fig.pdf}
    \caption{Left: Original \textit{Drosophila} data from
      \cite{Tan2009}. Right: After semi-supervised learning and
      classification, \cite{Breckels:2013}.}
  \end{figure}
\end{frame}


%-----------------------------------------------
% New section
%-----------------------------------------------

\section{Multi-localisation and uncertainly quantification}

\begin{frame}{How much do we learn? How much do we miss?}
  \begin{figure}
    \includegraphics[width=.8\linewidth]{./figs/preConcludePlot.png}
  \end{figure}
\end{frame}


\begin{frame}{A Bayesian Mixture Modelling Approach For Spatial Proteomics}

  \begin{itemize}

    \item<+-> \textit{T Augmented Gaussian Mixture model (TAGM)} is a
      \textbf{multivariate Gaussian generative model} for MS-based
      spatial proteomics data. It posits that each annotated
      sub-cellular niche can be modelled by a multivariate Gaussian
      distribution.

    \item<+-> With the prior knowledge that many proteins are not
      captured by known sub-cellular niches, we augment our model with
      an \textbf{outlier component}. Outliers are often dispersed and
      thus this additional component is described by a heavy-tailed
      distribution: the multivariate Student's t-distribution, leading
      us to a \textit{T Augmented Gaussian Mixture model}
      \citep{Crook:2018,Crook:2019}.

    \item<+-> This methodology allows proteome-wide
      \textbf{uncertainty quantification}, thus adding a further layer
      to the analysis of spatial proteomics.

  \end{itemize}
\end{frame}


\begin{frame}{}

    We initially model the distribution of profiles associated with
    proteins that localise to the $k$-th component as multivariate
    normal with mean vector $\boldsymbol{\mu}_k$ and covariance matrix
    $\Sigma_k$, so that:

    \begin{align}
      {\bf x}_i | z_i = k \quad \sim \mathcal{N}(\boldsymbol{\mu}_k, \Sigma_k) \label{equation::preq}
    \end{align}

    \pause

    We extend it by introducing an additional \textit{outlier
      component}. To do this, we augment our model by introducing a
    further indicator latent variable $\phi$. Each protein ${\bf x}_i$
    is now described by an additional variable $\phi_i$, with $\phi_i
    = 1$ indicating that protein ${\bf x}_i$ belongs to a organelle
    derived component and $\phi_i = 0$ indicating that protein ${\bf
      x}_i$ is not well described by these known components. This
    outlier component is modelled as a multivariate T distribution
    with degrees of freedom $\kappa$, mean vector $\bf{M}$, and scale
    matrix $V$.

    \begin{align}
      {\bf x}_i | z_i = k, \phi_i \quad \sim \mathcal{N}(\boldsymbol{\mu}_k, \Sigma_k)^{\phi_i}\mathcal{T}(\kappa, \boldsymbol{M}, V)^{1 - \phi_i }
    \end{align}


\end{frame}


\begin{frame}{}
      \begin{figure}
        \includegraphics[width=1\linewidth]{./figs/tagm_pca_res.pdf}
        \caption{Assignment of proteins of
          \textit{unknown} location to one of the annotated
          classes. The dots are scaled according to the protein
          assignment probabilities.}
      \end{figure}
\end{frame}

\begin{frame}{}
  \begin{figure}
    \includegraphics[width=.8\linewidth]{./figs/ConcludePlot.pdf}
  \end{figure}
\end{frame}

\begin{frame}
  \begin{figure}
    \centering
    \sidebysidecaption{0.55\linewidth}{0.42\linewidth}{
      \includegraphics[width=1\linewidth]{./figs/Q924C1-prob-1.pdf}
    }{
    \caption{\scriptsize \justifying Exportin 5 (Q924C1) forms part of
      the micro-RNA export machinery, transporting miRNA from the
      nucleus to the cytoplasm for further processing.  It then
      translocates back through the nuclear pore complex to return to
      the nucleus to mediate further transport between nucleus and
      cytoplasm. The model correctly infers that it most likely
      localises to the cytosol but there is some uncertainty with this
      assignment. This uncertainty is reflected in possible assignment
      of Exportin 5 to the nucleus non-chromatin and reflects the
      multi-location of the protein.}  }
    %% NOTE SVM failed to classify exportin 5 to any of the two
    %% biologically plausible locations, arguably due to the similarity
    %% of the cytosol and peroxysome, to which it got assigned.

  \end{figure}
\end{frame}


\begin{frame}{Whole sub-cellular proteome uncertainty}
  \begin{figure}
    \centering
    \includegraphics[width=.32\linewidth]{./figs/pca-tagm-mcmc-1.pdf}
    \includegraphics[width=.32\linewidth]{./figs/pca-tagm-map-1.pdf}
    \includegraphics[width=.32\linewidth]{./figs/prob-vs-shannon-1.pdf}
  \end{figure}
\end{frame}


%-----------------------------------------------
% New section
%-----------------------------------------------

\section{Spatial dynamics}


\begin{frame}{Spatial dynamics}

\end{frame}


%-----------------------------------------------
% New section
%-----------------------------------------------

\section{Behind the scences}


\begin{frame}{}
  \begin{center}
    \Large{\textbf{Behind the scenes}: software/data structures and
      open research practice.}
  \end{center}
\end{frame}


\begin{frame}{}

  Beyond the figures\footnote{... which are all reproducible, by the way.}

  \begin{itemize}
  \item<+-> Software: \textbf{infrastructure}
    (\href{http://bioconductor.org/packages/MSnbase}{\texttt{MSnbase}},
    \cite{Gatto:2012}), \textbf{dedicated machine learning}
    (\href{http://bioconductor.org/packages/pRoloc}{\texttt{pRoloc}},
    \cite{Gatto:2014a}), \textbf{interactive
      visualisation}\footnote{\url{https://lgatto.shinyapps.io/christoforou2015/}}
    (\href{http://bioconductor.org/packages/pRolocGUI}{\texttt{pRolocGUI}},
    \cite{pRolocGUI}) and \textbf{data}
    (\href{http://bioconductor.org/packages/pRolocdata}{\texttt{pRolocdata}},
    \cite{Gatto:2014a}) for spatial proteomics.
  \item<+-> The \href{http://bioconductor.org/}{\textbf{Bioconductor}}
    \citep{Huber:2015} ecosystem for high throughput biology data
    analysis and comprehension: \textbf{open source}, and
    \textbf{coordinated and collaborative\footnote{between and within
        domains/software} open development}, enabling
    \textbf{reproducible research}, enables understanding of the data
    (not a black box) and \textbf{drive scientific innovation}.
  \end{itemize}
\end{frame}

\begin{frame}{Open research: open source software}
  \centering
  \begin{figure}
  %% \includegraphics[width=\linewidth]{./figs_all/pRoloc_screen.png}
    \caption{\cite{Gatto:2014} Left: Public repository for the \texttt{pRoloc} software
      (\url{https://github.com/lgatto/pRoloc}). Right: offical
      Bioconductor page.}
  \end{figure}
\end{frame}

\begin{frame}{Open and reproducible research}
  \centering
  \begin{figure}
    %% \includegraphics[width=1\linewidth]{./figs_all/qsep_screen.png}
    \caption{\cite{Gatto:2018} reproducible document
      (\url{https://github.com/lgatto/QSep-manuscript}), preprint
      (\url{https://doi.org/10.1101/377630}) and paper
      (\url{https://doi.org/10.1016/j.cbpa.2018.11.015}).}
  \end{figure}
\end{frame}

%-----------------------------------------------
% New section
%-----------------------------------------------

\section{Conclusions}


\begin{frame}[fragile]{Conclusions}
  \begin{itemize}
  \item Protein sub-cellular localisation: technologies (hyperLOPIT)
    and opportunities.

  \item Reliance on computational biology, statistics and dedicated
    software (\texttt{pRoloc} \textit{et al.}) to interpret data and
    acquire biological knowledge.

  \item Rigorous computational infrastructure and sound data analysis
    and interpretation is a \textbf{long term investment}.

  \end{itemize}

\end{frame}



%-----------------------------------------------
% References
%-----------------------------------------------


\begin{frame}[allowframebreaks]{References}
  \scriptsize
  \bibliographystyle{plainnat}
  \bibliography{refs}
\end{frame}


%-----------------------------------------------
% Final slide
%-----------------------------------------------

\begin{frame}%[noframenumbering]
%\thispagestyle{empty}

% Logo UCL a gauche
\begin{tikzpicture}
  \useasboundingbox (0,0) rectangle(\the\paperwidth,1);
  \node[inner sep=0pt] at (1.7,2) {\includegraphics[width=.33\textwidth]{UCL_2018}};
 \end{tikzpicture}
\vspace{.1cm}



\begin{center}
  \textbf{Thank you for your attention}
\end{center}


\bigskip

Contact:

\begin{center}
  laurent.gatto@uclouvain.be – \url{lgatto.github.io/about}
\end{center}
 
\end{frame}


\end{document}
